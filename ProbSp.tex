\documentclass[journal,12pt,twocolumn]{IEEEtran}
\usepackage{amsthm}
\usepackage{graphics}
\usepackage{mathrsfs}
\usepackage{txfonts}
\usepackage{stfloats}
\usepackage{pgfplots}
\usepackage{cite}
\usepackage{cases}
\usepackage{mathtools}
\usepackage{caption}
\usepackage{enumerate}
\usepackage{enumitem}
\usepackage{amsmath}
\usepackage[utf8]{inputenc}
\usepackage[english]{babel}
\usepackage{multicol}
\usepackage{multirow}
\usepackage{longtable}
\usepackage{mathtools}
\usepackage{gensymb}
\usepackage{amssymb}
\usepackage{pgfplots}
\usepackage{hyperref}
\usepackage{listings}
\usepackage{color}
\usepackage{array}
\usepackage{calc}
\usepackage{ifthen}
\usepackage{hhline}


\title{Probability\&RV \\ Assignment-01}
\author{U Anuradha-ee21resch01008}
\date{\today}

\begin{document}

\maketitle

\section{Problem(9.3)}
random variable $z=n1-n2$, where $n1,n2\sim N(0,1)$
prove that z is a Gaussian random variable and find mean and variance of z.comment it.
\section{Solution}
Let x=n1 and  y=n2 then  z=x-y \\
Moment Generating Function (MGF)is given by 
\begin{equation}
   M_t(s)=E[e^{-st}] 
\end{equation}\\
pdf of x is 
\begin{equation}
    f_X(x)=\frac{1}{\sqrt{2\pi}\sigma_x}e^{-\frac{(x-\mu_x)^2}{2\sigma_x^2}}
\end{equation} \\
MGF of x is \\
$M_x(s)=\int_{-\infty}^{\infty}e^{-sx}f(x) dx $ \\
above equation represents Fourier Transform of f(x),we know that the F.T of $f_X(x)$ is
\begin{equation}
      M_x(s)=e^{-s\mu_x}e^{-\frac{s^2\sigma_x^2}{2}}
\end{equation}
similarly pdf of y is
\begin{equation}
    f_Y(y)=\frac{1}{\sqrt{2\pi}\sigma_y}e^{-\frac{(y-\mu_y)^2}{2\sigma_y^2}}
\end{equation}
MGF of $f_Y(y)$
\begin{equation}
    M_y(s)=e^{-s\mu_y}e^{-\frac{s^2\sigma_y^2}{2}}
\end{equation}
MGF of z can be written as \\
$M_z(s)=E[e^{-(x+y)s}]=M_x(s)\times M_y(s)$\\
by substituting eq(3) and eq(5) in $M_z(s)$\\
\begin{equation}
    M_z(s)=e^{-s(\mu_x+\mu_y)}e^{-\frac{s^2(\sigma_x^2+\sigma_y^2)}{2}}
\end{equation} 
\begin{equation}
   \therefore f_Z(z)=\frac{1}{\sqrt{2\pi}\sigma_z}e^{-\frac{(z-(\mu_x+\mu_y))^2}{2(\sigma_x^2+\sigma_y^2)}}
\end{equation}
lets substitute values $\mu_x=\mu_y=0$ and $\sigma_x^2=\sigma_y^2=1$ in above equations
\begin{equation}
    f_X(x)=\frac{1}{\sqrt{2\pi}}e^{-\frac{x^2}{2}}
\end{equation}
\begin{equation}
    f_Y(y)=\frac{1}{\sqrt{2\pi}}e^{-\frac{y^2}{2}}
\end{equation}
\begin{equation}
    M_x(s)=e^{-\frac{s^2}{2}}
\end{equation}
\begin{equation}
    M_y(s)=e^{-\frac{s^2}{2}}
\end{equation}
\begin{equation}
    M_z(s)=e^{-s^2}
\end{equation}\\
finally
\begin{equation}
  f_Z(z)=\frac{1}{\sqrt{2\pi}}e^{-\frac{z^2}{4}}
\end{equation}
\begin{tikzpicture}
   \begin{axis}[axis lines=left,xlabel={$x$},ylabel={$f_X(x)$}]
     \addplot[domain=-5:5, samples=20,color=red]{e^(-(x^2)/2)};
    \end{axis}
\end{tikzpicture}

\begin{tikzpicture}
   \begin{axis}[axis lines=left,xlabel={$y$},ylabel={$f_Y(y)$}]
     \addplot[domain=-5:5, samples=20,color=green]{e^(-(x^2)/2};
   \end{axis}
\end{tikzpicture}
\begin{tikzpicture}
 \begin{axis}[axis lines=left,xlabel={$z$},ylabel={$f_Z(z)$}]
   \addplot[domain=-5:5, samples=20,color=blue]{e^(-(x^2)/4)/2};
 \end{axis}
\end{tikzpicture}
\section{Conclusion}\\
Random variable which is either sum or difference of two standard normal variables is also a normal variable.\\
$ \mu_z=\mu_x+\mu_y=0$ and $\sigma_z^2=\sigma_x^2+\sigma_y^2=2$


\end{document}
